\documentclass[usenatbib,usegraphicx,letterpaper]{mn2e}
\usepackage[totalwidth=480pt,totalheight=680pt]{geometry}

\usepackage{amssymb}
\usepackage{epsfig}
\usepackage{amsmath}
\usepackage{color}

%\usepackage{hyperref}

%%%% Misc %%%
\newcommand{\beq}{\begin{equation}}
\newcommand{\eeq}{\end{equation}}
\newcommand{\beqray}{\begin{eqnarray}}
\newcommand{\eeqray}{\end{eqnarray}}

\newcommand{\ben}{\begin{enumerate}}
\newcommand{\een}{\end{enumerate}}
\newcommand{\bit}{\begin{itemize}}
\newcommand{\eit}{\end{itemize}}

%-------- journals
\newcommand{\araa}{ARAA~}
\newcommand{\apj}{ApJ~}
\newcommand{\apjl}{ApJL~}
\newcommand{\apjs}{ApJS~}
\newcommand{\mnras}{MNRAS~}
\newcommand{\nat}{Nature~}
\newcommand{\physrep}{Phys. Rep.~}
\newcommand{\aj}{AJ~}



\usepackage{epsfig}  \usepackage{graphicx}   \usepackage{rotating}

\begin{document}

\title[Forward-Modeling Satellite Mass Segregation]
{The Significance of Satellite Mass Segregation for Large-Scale Structure Cosmology}

\author[Hearin \& van den Bosch]{Andrew P. Hearin \& Frank C. van den Bosch}

\maketitle

\begin{abstract}
Dark matter subhalos are segregated by mass within their host halo. Mass segregation of {\em subhalos} is well-understood theoretically, but models for the strength of {\em satellite galaxy} mass segregation typically come from hydrodynamical simulations and complex semi-analytic models, in which mass segregation emerges as the combined effect of numerous parameters regulating distinct physical processes. Traditional formulations of the Halo Occupation Distribution (HOD) entirely neglect mass segregation, but here we show how it can be directly modeled using recently introduced HOD Decoration techniques. Our model spans the maximum allowable range of the effect while minimally expanding the parameter space. On scales of a few hundred kpc, mass segregation impacts galaxy clustering at the $X_1-X_2\%$ level, galaxy lensing at the $Y_1-Y_2\%$ level, and redshift-space distortions at the $Z_1-Z_2\%$ level. The potential impact of mass segregation on clustering and lensing becomes sub-percent on scales $r\gtrsim R_1$ kpc; for RSD statistics, the potential impact does not drop below percent-level until $r\gtrsim R_2$ Mpc. We make publicly available a {\tt Halotools}-based python implementation of the model. %Our code can be used together with modest-resolution N-body simulations to derive Bayesian posteriors on the strength of satellite galaxy mass segregation using forward-modeling techniques.  

\end{abstract}

\section{Introduction}
Some introduction goes here.

\section{Simulation}
We use  \citet{rockstar} subhalos. 

\bibliographystyle{mn2e} 
\bibliography{segmass}


\end{document}
